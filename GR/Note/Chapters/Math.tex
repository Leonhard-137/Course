\section{张量分析与微分几何基础}
  \subsection{流形和张量}
    \begin{definition}[光滑流形]
一个 $N$ 维光滑($C^\infty$ 可微)流形 $M$ 由以下条件定义:
\begin{enumerate}[label=(\arabic*)]
    \item $M$ 是一个拓扑流形
    \item 存在一族坐标卡 $\{(U_i, \varphi_i)\}$ 满足:
    \begin{itemize}
        \item $U_i$是开集且$\bigcup_i U_i = M$(覆盖条件)
        \item 对每个 $U_i$,$\varphi_i: U_i \to U_i' \subset \mathbb{R}^n$ 是同胚映射
    \end{itemize}
    \item 对任意非空交集 $U_i \cap U_j \neq \emptyset$,坐标变换映射
    \begin{align}
        \psi_{ij} = \varphi_i \circ \varphi_j^{-1}: \varphi_j(U_i \cap U_j) \to \varphi_i(U_i \cap U_j)
    \end{align}
    及其逆映射都是光滑的($C^\infty$)
\end{enumerate}
\end{definition}
\n{这些之乎者也的数学我也看不太明白,有空琢磨吧。但是大致可以看出流形满足两点,
\begin{enumerate}[label=(\arabic*)]
  \item 局部与$n$维欧几里得空间等价
  \item 由这些局域的欧几里得空间缝起来
\end{enumerate}}
坐标卡族 $\{(U_i, \varphi_i)\}$ 称为$M$的\textbf{图册}(Atlas),每个$U_i$称为\textbf{坐标邻域},
$\varphi_i$称为\textbf{坐标函数}。


\begin{example}[二维球面$S^2$]
拿地球表面做例子,$S^2$是一个二维流形,因为:
\begin{itemize}
    \item 没有一张地图能覆盖地球上所有地方,世界地图也有地方被剪开。
    \item 给每个地区造一个地图,拼起来可以覆盖全球,且局部看起来都像平面。
    \item 地图与地图之间同一个位置可以转换坐标。
\end{itemize}
这样看起来,取的名字坐标卡和图册还是挺形象的。
\end{example}
\textbf{坐标变换}:设点 $p$ 在坐标卡 $(U, x^\mu)$ 下的坐标为 $x^\mu$,在另一坐标卡 $(V, \tilde x^\mu)$ 下的坐标为 $\tilde x^\mu$。则在交集 $U \cap V$ 上,$\tilde x^\mu = \tilde x^\mu(x)$,可以给出,
\begin{align}
  \tilde{\partial_\mu} :&= \frac{\partial}{\partial x^\mu} = \frac{\partial x^\nu}{\partial \tilde x^\mu} \frac{\partial}{\partial x^\nu}\\
  d\tilde x^\mu &= \frac{\partial \tilde x^\mu}{\partial x^\nu} \, dx^\nu
\end{align}

\subsubsection*{流形上的张量}

\begin{definition}[标量场]
  标量场是流形到实数的映射 \( f: M \to \mathbb{R} \)。它是一个零阶张量场,即在任何坐标变换下,其值不变:
  \begin{align}
    \tilde{f}(\tilde{x}) = f(x)
  \end{align}
\end{definition}
\n{这个定义,说人话就是$T^{\mu}{}_{\nu}A_{\mu}B^{\nu}$是个实数,张量把向量空间的元素(B)和对偶空间的元素(A)映射到实数上去了。}

\begin{definition}[张量场的定义]
  一个 \( (p, q) \) 型张量场 \( T \) 是一个多重线性映射,它将 \( p \) 个对偶空间元素(协变向量)和 \( q \) 个向量空间元素(逆变向量)映射到实数:
  \begin{align}
    T: (T^*_p M)^{\otimes p} \times (T_p M)^{\otimes q} \to \mathbb{R}.
  \end{align}
\end{definition}
\n{上面的定义很数学,但是课程中给的张量定义其实就简单的是满足张量变换性质的量,不同教科书也有不同的定义,判定张量还是用坐标变换性质。}
在坐标变换下,张量场的分量变换遵循以下规则:
\begin{align}
  T^{\mu_1 \dots \mu_p}_{\phantom{\mu_1 \dots \mu_p} \nu_1 \dots \nu_q}(x)
  &= \frac{\partial x^{\mu_1}}{\partial \tilde{x}^{\alpha_1}} \cdots \frac{\partial x^{\mu_p}}{\partial \tilde{x}^{\alpha_p}} \frac{\partial \tilde{x}^{\beta_1}}{\partial x^{\nu_1}} \cdots \frac{\partial \tilde{x}^{\beta_q}}{\partial x^{\nu_q}} 
  \tilde{T}^{\alpha_1 \dots \alpha_p}_{\phantom{\alpha_1 \dots \alpha_p} \beta_1 \dots \beta_q}(\tilde{x}),
\end{align}
其中 \( \tilde{x} \) 是新的坐标系下的坐标,\( x \) 是旧坐标系下的坐标,变换矩阵(即偏导数)对应了坐标变换规则。


    \subsubsection*{张量的运算和对称性}

      \begin{itemize}
        \item \textbf{加减}:$C^{\mu}{}_{\nu}(x) = A^{\mu}{}_{\nu}(x) + B^{\mu}{}_{\nu}(x)$ \\
              同类型(即同阶且指标特性相同)的张量可以相加减。
        \item \textbf{乘法(直乘)}:$C^\mu{}_{\nu} = A^\mu B_\nu$ \\ 两个张量可以相乘,结果是一个新张量,其阶数为原张量阶数之和。
        \item \textbf{缩并}:$T^{\mu}{}_{\nu \lambda}$\\对一个张量的一个上指标和一个下指标进行求和,从而降低张量的阶数,张量缩并之后仍然是张量。
        \item \textbf{商定理}:设在某坐标系中有张量关系
          \begin{align}
          A^\mu(x) &= B^\mu{}_{\alpha}(x)\,C^{\alpha}(x),
          \end{align}
        其中 $A^\mu$ 为 $(1,0)$ 张量、$B^\mu{}_{\alpha}$ 为 $(1,1)$ 张量。
        证明 $C^{\alpha}$ 为 $(1,0)$ 张量,即在坐标变换 $x\mapsto x'$ 下满足
          \begin{align}
          C'^{\,\alpha}(x') &= \frac{\partial x'^{\alpha}}{\partial x^\beta}C^\beta(x)
          \end{align}
      \end{itemize}

    \subsubsection*{张量的对称性}
      \begin{itemize}
        \item \textbf{对称张量}:对于二阶张量 \( T_{\mu \nu} \),若 \( T_{\mu \nu} = T_{\nu \mu} \),则称其为对称张量。
        \item \textbf{反对称张量}:若 \( T_{\mu \nu} = -T_{\nu \mu} \),则称其为反对称张量。
        \item \textbf{对称分解}:任何二阶张量 \( T^{\mu \nu} \) 都可以唯一地分解为对称部分 \( S^{\mu \nu} \) 和反对称部分 \( A^{\mu \nu} \) 之和:
        \begin{align}
          T^{\mu \nu} &= S^{\mu \nu} + A^{\mu \nu} \\
          S^{\mu \nu} &= \frac{1}{2}(T^{\mu \nu} + T^{\nu \mu}) =: T^{(\mu \nu)} \\
          A^{\mu \nu} &= \frac{1}{2}(T^{\mu \nu} - T^{\nu \mu}) =: T^{[\mu \nu]}
        \end{align}
      \end{itemize}

\subsubsection*{反对称张量的常用性质}

\begin{enumerate}[label=(\arabic*)]
  \item 反对称指标取值相同时,该分量为零。反对称指标与对称指标的缩并为零。
  
  \begin{proof}
  设$T_{\mu\nu} = -T_{\nu\mu}$,则当$\mu = \nu$时:
  \begin{align}
  T_{\mu\mu} = -T_{\mu\mu} \Rightarrow 2T_{\mu\mu} = 0 \Rightarrow T_{\mu\mu} = 0
  \end{align}
  设$S^{\mu\nu}$对称,则:
  \begin{align}
  T_{\mu\nu}S^{\mu\nu} = -T_{\nu\mu}S^{\mu\nu} = -T_{\mu\nu}S^{\nu\mu} = -T_{\mu\nu}S^{\mu\nu} \Rightarrow T_{\mu\nu}S^{\mu\nu} = 0
  \end{align}
  \end{proof}

  \item $n$维流形上最高阶的反对称张量为$n$阶,且该张量只有一个独立分量。
  
  \begin{proof}
  $n$阶反对称张量的独立分量数为组合数:
  \begin{align}
  C_n^n = \frac{n!}{n!0!} = 1
  \end{align}
  可表示为$T_{[\mu_1\cdots\mu_n]} = c\epsilon_{\mu_1\cdots\mu_n}$,其中$c$为常数。
  \end{proof}

  \item $n$维流形上的$n-1$阶反对称张量有$n$个独立分量。
  
  \begin{proof}
  独立分量数为组合数:
  \begin{align}
  C_{n-1}^n = \frac{n!}{(n-1)!1!} = n
  \end{align}
  例如三维流形上二阶反对称张量有$C_2^3 = 3$个独立分量。
  \end{proof}

  \item 一般地,$n$维流形上的$m$阶反对称张量有$C_m^n$个独立分量。
  
  \begin{proof}
  反对称性要求指标互不相同,独立分量数等于从$n$个指标中选$m$个的组合数:
  \begin{align}
  C_m^n = \frac{n!}{m!(n-m)!}
  \end{align}
  \end{proof}
\end{enumerate}

\subsubsection*{张量的平移和仿射联络}

在平直时空中,矢量平移很简单——只需保持分量不变,所以虽然矢量是逐点定义的,仍然可以直接把相邻两点的矢量相减取导数。但一个在球表面的矢量,如果它保持与切面的夹角不变,沿着某路径平移(这就是平行移动的直观),得到的矢量分量与原来的矢量发生了变换,这就不能直接做差了,必须讨论矢量 $A^{\mu}(P)$ 和矢量 $A^{\mu}(P\rightarrow Q)$ 之间的关系,后者表示把P处的矢量平移到Q处。为了解决这个问题,引入了\textbf{仿射联络} $\Gamma^{\lambda}{}_{\mu\nu}$,它定义了如何在流形上"平移"矢量。
\begin{definition}[协变矢量的平移]
协变矢量 $A_\mu$ 从点 $P$ 沿无穷小位移 $dx^\nu$ 平移到 $Q$ 点的规则为:
\begin{align}
A_\mu(P \to Q) = A_\mu(P) + \Gamma^{\lambda}{}_{\mu\nu}(P)\, A_\lambda(P)\, dx^\nu
\end{align}
\end{definition}

\begin{definition}[逆变矢量的平移]
逆变矢量 $B^\mu$ 的平移规则为:
\begin{align}
B^\mu(P \to Q) = B^\mu(P) - \Gamma^{\mu}{}_{\nu\lambda}(P)\, B^\nu(P)\, dx^\lambda
\end{align}
\end{definition}

\n{这里也可以看出两种张量定义的区别,按那个繁杂数学定义的话,联络就是张量,但是这里采用坐标变换的定义,联络不是张量。}
\begin{theorem}[仿射联络变换性质]
仿射联络 $\Gamma^{\lambda}{}_{\mu\nu}$ 的坐标变换规则为:
\begin{align}
\tilde{\Gamma}^{\lambda}{}_{\mu\nu}(\tilde{x}) =
\frac{\partial^2 x^\alpha}{\partial \tilde{x}^\nu \partial \tilde{x}^\mu}
\frac{\partial \tilde{x}^\lambda}{\partial x^\alpha}
+ \Gamma^{\alpha}{}_{\rho\beta}(x)\,
\frac{\partial x^\rho}{\partial \tilde{x}^\mu}
\frac{\partial x^\beta}{\partial \tilde{x}^\nu}
\frac{\partial \tilde{x}^\lambda}{\partial x^\alpha}
\end{align}
仿射联络不是张量,但是不难验证仿射联络的差是张量。
\end{theorem}

\begin{derivationnote}[仿射联络的变换性质]
张量的平移操作可以把 $P$ 点(坐标为 $x^\mu$)的张量平移到邻近的 $Q$ 点(坐标为 $x^\mu + dx^\mu$)并成为 $Q$ 点的张量。为了实现这一操作,需要引进一种特殊的场,称为仿射联络。以协变矢量为例,将 $P$ 点的协变矢量 $A_\mu(P)$ 平移至 $Q$ 点后的矢量记作 $A_\mu(P \to Q)$。线性理论的要求给出其定义:
\begin{align}
A_{\mu}(P \rightarrow Q) - A_{\mu}(P)
= \Gamma^{\lambda}{}_{\mu\nu}(P)\, A_{\lambda}(P)\, d x^{\nu}
\label{eq:parallel_transport_def}
\end{align}
其中的比例系数 $\Gamma^{\lambda}{}_{\mu\nu}(P)$ 就叫 $P$ 点的仿射联络。

要求 $A_\mu(P \to Q)$ 是 $Q$ 点的协变矢量,即在新坐标系 $\{\tilde{x}^\mu\}$ 下,它应满足张量变换规则:
\begin{align}
\widetilde{A}_{\mu}(P \rightarrow Q) = \left( \frac{\partial x^{\alpha}}{\partial \tilde{x}^{\mu}} \right)_{Q} A_{\alpha}(P \rightarrow Q)
\label{eq:tensor_transformation_requirement}
\end{align}
在新坐标系中,平移规则同样定义为:
\begin{align}
\widetilde{A}_{\mu}(P \rightarrow Q) - \widetilde{A}_{\mu}(P)
= \tilde{\Gamma}^{\lambda}{}_{\mu\nu}(P)\, \widetilde{A}_{\lambda}(P)\, d\tilde{x}^{\nu}
\label{eq:parallel_transport_tilde}
\end{align}

将定义式 \eqref{eq:parallel_transport_def} 和 \eqref{eq:parallel_transport_tilde} 代入要求条件 \eqref{eq:tensor_transformation_requirement},得到关系式:
\begin{align}
\widetilde{A}_{\mu}(P) + \tilde{\Gamma}^{\lambda}{}_{\mu\nu}(P)\,\widetilde{A}_{\lambda}(P)\, d\tilde{x}^{\nu}
= \left( \frac{\partial x^{\alpha}}{\partial \tilde{x}^{\mu}} \right)_{Q}
\left[ A_{\alpha}(P) + \Gamma^{\lambda}{}_{\alpha\nu}(P)\, A_{\lambda}(P)\, d x^{\nu} \right]
\label{eq:master_relation}
\end{align}

坐标变换矩阵在 $P$ 和 $Q$ 点存在微小差异,其一阶展开为:
\begin{align}
\left( \frac{\partial x^{\alpha}}{\partial \tilde{x}^{\mu}} \right)_{Q}
= \left( \frac{\partial x^{\alpha}}{\partial \tilde{x}^{\mu}} \right)_{P}
+ \left( \frac{\partial^{2} x^{\alpha}}{\partial \tilde{x}^{\nu} \partial \tilde{x}^{\mu}} \right)_{P} d \tilde{x}^{\nu}
\label{eq:expansion_of_jacobian}
\end{align}
同时,$P$ 点矢量的变换规则及坐标微分的变换规则为:
\begin{align}
\widetilde{A}_\mu(P) &= \left( \frac{\partial x^{\alpha}}{\partial \tilde{x}^{\mu}} \right)_P A_\alpha(P) \\
d x^{\nu} &= \frac{\partial x^{\nu}}{\partial \tilde{x}^{\beta}} d\tilde{x}^{\beta}
\label{eq:auxiliary_transformations}
\end{align}

将式 \eqref{eq:expansion_of_jacobian} 和 \eqref{eq:auxiliary_transformations} 代入主关系式 \eqref{eq:master_relation},并利用 $\widetilde{A}_\mu(P)$ 的表达式,可得:
\begin{align}
\left( \frac{\partial x^{\alpha}}{\partial \tilde{x}^{\mu}} \right)_P &A_\alpha(P)
+ \tilde{\Gamma}^{\lambda}{}_{\mu\nu}(P)
\left( \frac{\partial x^{\beta}}{\partial \tilde{x}^{\lambda}} \right)_P A_\beta(P)\, d\tilde{x}^{\nu}
= \nonumber \\
&\left[ \left( \frac{\partial x^{\alpha}}{\partial \tilde{x}^{\mu}} \right)_P
+ \left( \frac{\partial^{2} x^{\alpha}}{\partial \tilde{x}^{\nu} \partial \tilde{x}^{\mu}} \right)_P d \tilde{x}^{\nu} \right] A_\alpha(P) \nonumber \\
&+ \left( \frac{\partial x^{\alpha}}{\partial \tilde{x}^{\mu}} \right)_P
\Gamma^{\lambda}{}_{\alpha\beta}(P)\, A_\lambda(P)\, \frac{\partial x^\beta}{\partial \tilde{x}^\nu} d\tilde{x}^\nu
\end{align}

等式两边同时减去公共项 $\left( \frac{\partial x^{\alpha}}{\partial \tilde{x}^{\mu}} \right)_P A_\alpha(P)$,并略去二级无穷小量,整理后比较 $d\tilde{x}^{\nu}$ 和 $A_\alpha(P)$ 的系数(利用其任意性),可得:
\begin{align}
\tilde{\Gamma}^{\lambda}{}_{\mu\nu}(P)\, \frac{\partial x^{\gamma}}{\partial \tilde{x}^{\lambda}}
= \frac{\partial^{2} x^{\gamma}}{\partial \tilde{x}^{\nu} \partial \tilde{x}^{\mu}}
+ \Gamma^{\gamma}{}_{\rho\beta}(P)\,
\frac{\partial x^{\rho}}{\partial \tilde{x}^{\mu}} \frac{\partial x^{\beta}}{\partial \tilde{x}^{\nu}}
\label{eq:intermediate_result}
\end{align}

为解出 $\tilde{\Gamma}^{\lambda}{}_{\mu\nu}$,将式 \eqref{eq:intermediate_result} 两边同时乘以 $\frac{\partial \tilde{x}^{\lambda}}{\partial x^{\gamma}}$ 并对 $\gamma$ 求和,利用关系式 $\frac{\partial x^{\gamma}}{\partial \tilde{x}^{\lambda}} \frac{\partial \tilde{x}^{\lambda}}{\partial x^{\gamma'}} = \delta^{\gamma}_{\gamma'}$,最终得到仿射联络的坐标变换公式:
\begin{align}
\tilde{\Gamma}^{\lambda}{}_{\mu\nu}(\tilde{x}) =
\frac{\partial^{2} x^{\alpha}}{\partial\tilde{x}^{\nu}\partial\tilde{x}^{\mu}}
\frac{\partial\tilde{x}^{\lambda}}{\partial x^{\alpha}}
+ \Gamma^{\alpha}{}_{\rho\beta}(x)\,
\frac{\partial x^{\rho}}{\partial\tilde{x}^{\mu}}
\frac{\partial x^{\beta}}{\partial\tilde{x}^{\nu}}
\frac{\partial\tilde{x}^{\lambda}}{\partial x^{\alpha}}
\label{eq:affine_connection_transformation}
\end{align}

反过来,只要仿射联络按式 \eqref{eq:affine_connection_transformation} 的规则变换,就可以保证 $A_\mu(P \to Q)$ 是 $Q$ 点的协变矢量。因此,仿射联络的此变换性质是保证平移后矢量仍具协变性的充要条件。
\end{derivationnote}

\begin{theorem}[零联络坐标定理]
在流形 $M$ 上,任取一点 $P \in M$,总存在一组局部坐标系 $\{x^\mu\}$,使得在 $P$ 点联络系数 $\Gamma^{\lambda}{}_{\mu\nu}(P) = 0$。
\end{theorem}
\n{证明还没看懂,主要是有一个小的展开不知道是什么意思,待补充。}
\begin{derivationnote}[零联络坐标定理和等效原理]
在局部坐标系中,若联络系数 $\Gamma^{\lambda}{}_{\mu\nu}(P) = 0$,则在 $P$ 点的切空间中,任意两个切向量的平行移动不受曲率影响,具有良好的物理意义,而度规张量总可以做一个合同变换使其变为 Minkowski 度规,这与等效原理相符,即在小范围内,任何引力场都可以被视为局部的惯性参考系。
\end{derivationnote}

\subsubsection*{普通导数不是张量}

在曲率时空或非线性坐标系中,普通偏导数 $\partial_\mu$ 对张量的作用不再保持协变性,这是引入协变导数的根本原因。

设 $A^{\mu}$ 为逆变矢量场,其在坐标变换 $x^{\mu} \to x'^{\mu}$ 下的变换律为
\begin{align}
A'^{\mu} = \frac{\partial x'^{\mu}}{\partial x^{\nu}} A^{\nu}.
\end{align}

我们希望考察 $\partial_\lambda A^{\mu}$ 是否也是张量。对上式两边求偏导:
\begin{align}
\partial'_{\sigma} A'^{\mu}
= \frac{\partial}{\partial x'^{\sigma}}
\left( \frac{\partial x'^{\mu}}{\partial x^{\nu}} A^{\nu} \right)
= \frac{\partial x^{\lambda}}{\partial x'^{\sigma}}
\frac{\partial}{\partial x^{\lambda}}
\left( \frac{\partial x'^{\mu}}{\partial x^{\nu}} A^{\nu} \right).
\end{align}

展开导数:
\begin{align}
\partial'_{\sigma} A'^{\mu}
= \frac{\partial x^{\lambda}}{\partial x'^{\sigma}}
\left[
\frac{\partial^2 x'^{\mu}}{\partial x^{\lambda} \partial x^{\nu}} A^{\nu}
+ \frac{\partial x'^{\mu}}{\partial x^{\nu}} \partial_{\lambda} A^{\nu}
\right].
\end{align}

若 $\partial_\lambda A^{\mu}$ 真是张量,应满足:
\begin{align}
\partial'_{\sigma} A'^{\mu}
\stackrel{?}{=}
\frac{\partial x'^{\mu}}{\partial x^{\rho}}
\frac{\partial x^{\lambda}}{\partial x'^{\sigma}}
\partial_{\lambda} A^{\rho}.
\end{align}

然而比较可见,真实结果多出一项:
\begin{align}
\color{red}{
\frac{\partial x^{\lambda}}{\partial x'^{\sigma}}
\frac{\partial^2 x'^{\mu}}{\partial x^{\lambda} \partial x^{\nu}} A^{\nu},
}
\end{align}
该项一般不为零,因此,普通偏导数 $\partial_\lambda A^{\mu}$ 在一般坐标变换下并不满足张量变换律:

\begin{align}
\boxed{
\partial_\lambda A^{\mu} \text{ 不是张量。}
}
\end{align}

\begin{derivationnote}[几何与物理解释]
在曲率空间或非直角坐标中,坐标基底 $\mathbf{e}_{\mu}$ 自身随位置变化,
而普通偏导 $\partial_\mu A^{\nu}$ 只考虑了分量的变化,未考虑基底的变化。
因此它描述的“变化”依赖于坐标选择,并非几何不变的物理量。
\end{derivationnote}

\begin{example}[极坐标]
考虑平面上的矢量场 $\mathbf{A} = A^r \mathbf{e}_r + A^\theta \mathbf{e}_\theta$,  
其中极坐标基矢 $\mathbf{e}_r$ 与 $\mathbf{e}_\theta$ 均随角度 $\theta$ 变化:
\begin{align}
\frac{\partial \mathbf{e}_r}{\partial \theta} = \mathbf{e}_\theta, 
\qquad
\frac{\partial \mathbf{e}_\theta}{\partial \theta} = -\mathbf{e}_r.
\end{align}

若取普通偏导:
\begin{align}
\frac{\partial \mathbf{A}}{\partial \theta}
= \frac{\partial A^r}{\partial \theta} \mathbf{e}_r
+ \frac{\partial A^\theta}{\partial \theta} \mathbf{e}_\theta
+ A^r \frac{\partial \mathbf{e}_r}{\partial \theta}
+ A^\theta \frac{\partial \mathbf{e}_\theta}{\partial \theta}.
\end{align}

代入上式得:
\begin{align}
\frac{\partial \mathbf{A}}{\partial \theta}
= \left( \frac{\partial A^r}{\partial \theta} - A^\theta \right) \mathbf{e}_r
+ \left( \frac{\partial A^\theta}{\partial \theta} + A^r \right) \mathbf{e}_\theta.
\end{align}
若我们只取分量的偏导 $\partial_\theta A^r$、$\partial_\theta A^\theta$,
则忽略了基底旋转的贡献,结果无法表示真实的几何变化,
也就是说此时 $\partial_\theta A^\mu$ 的组合不再变换为一个真正的张量。
\end{example}

  \subsection{协变导数(Covariant Derivative)}
解决平移的问题之后就可以定义导数,利用Q点的逆变矢量 $A^{\mu}(Q)$ 和平移后的矢量 $A^{\mu}(P \to Q)$ 做差,再除以位移 $dx^\nu$,取极限 $dx^\nu \to 0$就可以得到协变导数,也就是:
\begin{align}
  \lim_{Q \to P} \frac{A^{\mu}(Q) - A^{\mu}(P \to Q)}{dx^\nu}
\end{align}
不难看出,这个导数衡量的时逆变矢量在流形上平移的变化率。
\begin{definition}[协变导数的定义]
设 $A^{\mu}$ 为\textbf{逆变矢量},则协变导数定义为
\begin{align}
D_{\nu} A^{\mu} \equiv \partial_{\nu} A^{\mu} + \Gamma^{\mu}{}_{\lambda \nu} A^{\lambda},
\end{align}
其中 $\Gamma^{\mu}{}_{\lambda \nu}$ 为 \textbf{Christoffel 符号}(联络系数)。
\end{definition}

对于协变\textbf{矢量} $B_{\mu}$,有
\begin{align}
D_{\nu} B_{\mu} = \partial_{\nu} B_{\mu} - \Gamma^{\lambda}{}_{\mu \nu} B_{\lambda}.
\end{align}

若 $\Gamma^{\rho}{}_{\mu\nu}$ 为 $(1,2)$ 阶张量,则其协变导数为
\begin{align}
D_{\lambda} \Gamma^{\rho}{}_{\mu\nu}
= \partial_{\lambda} \Gamma^{\rho}{}_{\mu\nu} \\
+ \Gamma^{\rho}{}_{\sigma \lambda} \Gamma^{\sigma}{}_{\mu\nu}
- \Gamma^{\sigma}{}_{\mu \lambda} \Gamma^{\rho}{}_{\sigma\nu}
- \Gamma^{\sigma}{}_{\nu \lambda} \Gamma^{\rho}{}_{\mu\sigma}.
\end{align}


\subsubsection*{基本性质}

协变导数满足以下性质:

\begin{enumerate}
  \item \textbf{线性性:}
  \begin{align}
  D_{\nu} (A + B) = D_{\nu} A + D_{\nu} B.
  \end{align}

  \item \textbf{Leibniz 法则:}
  对任意张量积有
  \begin{align}
  D_{\nu} (A \otimes B) = (D_{\nu} A) \otimes B + A \otimes (D_{\nu} B).
  \end{align}

  \item \textbf{与坐标变换相容性:}
  若 $T$ 为张量,则 $D_{\nu} T$ 亦为张量。

  \item \textbf{与度规张量的相容性:}
  对度规张量 $g_{\mu\nu}$,有
  \begin{align}
  D_{\lambda} g_{\mu\nu} = 0.
  \end{align}
  该条件称为\textbf{度规相容性条件}(metric compatibility),意味着在平行移动过程中矢量长度保持不变。
  
  \item \textbf{对标量场作用等同于普通导数:}
  若 $f$ 为标量场,则
  \begin{align}
  D_{\mu} f = \partial_{\mu} f.
  \end{align}
\end{enumerate}

\begin{derivationnote}[推广到高阶张量场]
对任意 $(r,s)$ 型张量 $T^{\mu_1\cdots \mu_r}{}_{\nu_1\cdots\nu_s}$,协变导数为
\begin{align}
D_{\lambda} T^{\mu_1\cdots \mu_r}{}_{\nu_1\cdots\nu_s} \\
&= \partial_{\lambda} T^{\mu_1\cdots \mu_r}{}_{\nu_1\cdots\nu_s}
+ \sum_{i=1}^{r} \Gamma^{\mu_i}{}_{\sigma \lambda} T^{\mu_1\cdots \sigma \cdots \mu_r}{}_{\nu_1\cdots\nu_s}
- \sum_{j=1}^{s} \Gamma^{\sigma}{}_{\nu_j \lambda} T^{\mu_1\cdots\mu_r}{}_{\nu_1\cdots \sigma \cdots \nu_s}.
\end{align}
该公式体现出:对每个上标加一个 $\Gamma$ 项,对每个下标减一个 $\Gamma$ 项。
\end{derivationnote}


\subsection{度规张量(Metric Tensor)}

\begin{definition}[度规张量]
  我们称满足以下性质的二阶协变张量为\textbf{度规张量} $g_{\mu\nu}$:
  \begin{enumerate}[label=(\arabic*)]
    \item \textbf{对称性:} 
    \begin{align}
    g_{\mu\nu} = g_{\nu\mu}.
    \end{align}

    \item \textbf{非退化性:}
    \begin{align}
    \det(g_{\mu\nu}) \neq 0,
    \end{align}
    因此存在逆度规 $g^{\mu\nu}$,满足
    \begin{align}
    g^{\mu\lambda} g_{\lambda\nu} = \delta^{\mu}_{\nu}.
    \end{align}

    \item \textbf{Lorentz 号差:} 
    在广义相对论中度规符号取 $(-,+,+,+)$,即具有一个负特征值与三个正特征值。
  \end{enumerate}

\end{definition}
\n{对号差的理解:\\
坐标变换对度规矩阵来说是一次合同变换,根据 Sylvester 惯性定理,合同变换不会改变矩阵的正负特征值个数。
因此度规的符号型(signature)在坐标变换下保持不变。\\
不同教材对号差的定义略有不同,有的定义为“负特征值的个数”,有的定义为“正负数差”,但物理意义相同。}

\subsubsection{Levi--Civita 联络与度规相容性}

在广义相对论中采用的联络为\textbf{Levi--Civita 联络},它由以下两条条件唯一确定:

\begin{align}
D_{\lambda} g_{\mu\nu} = 0 
\quad &\text{(度规相容性条件)}, \label{eq:metric-compat}\\[4pt]
\Gamma^{\rho}{}_{\mu\nu} = \Gamma^{\rho}{}_{\nu\mu}
\quad &\text{(无挠性条件)}. \label{eq:torsion-free}
\end{align}

度规相容性意味着在平行移动下矢量的长度和角度保持不变;
无挠性意味着联络的对称性(即不存在“平移旋转”扭曲)。

\begin{theorem}[Riemannian 几何基本定理]
Levi--Civita 联络的无挠 Christoffel 符号由度规唯一确定:
\begin{align}\label{eq:christoffel}
\Gamma^{\lambda}{}_{\mu\nu}
= \tfrac{1}{2} g^{\lambda\sigma}
\left(
\partial_{\mu} g_{\nu\sigma}
+ \partial_{\nu} g_{\mu\sigma}
- \partial_{\sigma} g_{\mu\nu}
\right).
\end{align}
\end{theorem}

\begin{proof}
由度规相容条件 \eqref{eq:metric-compat} 展开得:
\begin{align}
\partial_{\lambda} g_{\mu\nu}
- \Gamma^{\sigma}{}_{\lambda\mu} g_{\sigma\nu}
- \Gamma^{\sigma}{}_{\lambda\nu} g_{\mu\sigma} = 0.
\end{align}
对指标 $(\lambda, \mu, \nu)$ 进行循环排列并线性组合三式,解出 $\Gamma^{\lambda}{}_{\mu\nu}$,即可得到 \eqref{eq:christoffel}。
\end{proof}

\n{公式 \eqref{eq:christoffel} 表明联络完全由度规决定,因此一旦选定度规,空间的几何性质也就被确定。}

\subsubsection{度规与因果结构}

在流形上,度规 $g_{\mu\nu}$ 定义了时空间隔:
\begin{align}
ds^2 = g_{\mu\nu}\,dx^{\mu} dx^{\nu}.
\end{align}

\n{不同于欧几里得空间,洛伦兹度规并非正定,因此可以定义“时间类”、“光类”、“空间类”三类向量。}

对任意向量 $U^{\mu}$,
\begin{align}
g(U,U) = g_{\mu\nu} U^{\mu} U^{\nu}
\begin{cases}
>0, & \text{时类(timelike)},\\
=0, & \text{光类(null / lightlike)},\\
<0, & \text{空类(spacelike)}.
\end{cases}
\end{align}

这一定义在每个切空间上确定了局域的\textbf{因果结构}:
时类方向对应有质量粒子的世界线,光类方向定义光锥,
空类方向代表纯空间位移。

若存在一组基矢 $\{ e_a \}$ 使得
\begin{align}
g_{\mu\nu} e_a^{\mu} e_b^{\nu} = \eta_{ab},
\end{align}
其中 $\eta_{ab} = \mathrm{diag}(-1,1,1,1)$,则称 $\{ e_a \}$ 为 $g_{\mu\nu}$ 的\textbf{正交归一标架(orthonormal frame)}。

\subsection{平行移动与测地线}
前面由仿射联络定义的平行移动是在临近点的移动,但是要保持矢量在一整条路径上的移动都是平移就不能用那个定义了。
想保证这一点只要保证矢量与切矢量夹角不变就行了,度规相容性又保证了矢量在平行移动时长度不变,因此只要保证矢量在切矢量上的夹角不变就行了。
设曲线参数为 $\lambda$,曲线为 $x^{\mu}(\lambda)$,
切向量
\begin{align}
v^{\mu} \equiv \frac{dx^{\mu}}{d\lambda}.
\end{align}

沿曲线的任意向量场 $A^{\nu}(\lambda)$ 的协变导数定义为
\begin{align}
\frac{D A^{\nu}}{D\lambda}
\coloneqq v^{\mu} D_{\mu} A^{\nu}
= \frac{dA^{\nu}}{d\lambda}
+ \Gamma^{\nu}{}_{\alpha\beta}\,v^{\alpha} A^{\beta}.
\end{align}

若 $\dfrac{D A^{\nu}}{D\lambda}=0$,称 $A^{\nu}$ 沿曲线\textbf{平行移动}。

\n{这里的“平行移动”是指在同一条曲线上逐点比较向量,强调基底的变化。}

\subsubsection*{测地线作为“切向量的平行移动”}

若取被平行移动的向量为曲线的切向量自身 $A^{\nu}=v^{\nu}$,有:
\begin{align}
\frac{D v^{\nu}}{D\lambda} 
= v^{\mu} D_{\mu} v^{\nu} = 0,
\end{align}
即
\begin{align}\label{eq:geodesic}
\frac{d^2 x^{\nu}}{d\lambda^2}
+ \Gamma^{\nu}{}_{\alpha\beta}
\frac{dx^{\alpha}}{d\lambda}\frac{dx^{\beta}}{d\lambda} = 0.
\end{align}

这就是\textbf{测地线方程}。  
几何意义:测地线是沿自身方向被平行移动的曲线,
物理意义:自由粒子在无外力作用下沿测地线运动。

\subsubsection*{关于参数选择(仿射参数)}

若曲线参数 $\lambda$ 不是仿射参数,则有
\begin{align}
\frac{D v^{\nu}}{D\lambda} = \kappa(\lambda) v^{\nu},
\end{align}
其中 $\kappa(\lambda)$ 是某个标量函数。
通过重参数化可令 $\kappa=0$,
从而恢复标准形式 \eqref{eq:geodesic}。
此时的 $\lambda$ 即为\textbf{仿射参数(affine parameter)}。


  \subsection{曲率张量与 Bianchi 恒等式}

  \begin{definition}{曲率张量}
    在无挠、度规相容的 Levi--Civita 联络下,\textbf{协变导数的对易子}作用在逆变矢量 $V^\rho$ 上定义\textbf{Riemann 曲率张量}:
  \begin{equation}\label{eq:comm-def}
  [\nabla_\mu,\nabla_\nu]\,V^\rho
  \;=\; R^{\rho}{}_{\sigma\mu\nu}\,V^\sigma.
  \end{equation}
  相应地,对协变矢量 $\omega_\rho$ 有
  \begin{equation}
  [\nabla_\mu,\nabla_\nu]\,\omega_\rho
  \;=\; -\,R^{\sigma}{}_{\rho\mu\nu}\,\omega_\sigma.
  \end{equation}
  对一般 $(r,s)$ 张量,$R$ 对每个指标线性作用(逆变指标为 $+R$,协变指标为 $-R$)。

  \end{definition}
  
  \paragraph{坐标表达式}
  由 \eqref{eq:comm-def} 可得标准分量式
  \begin{equation}\label{eq:riemann-gamma}
  R^{\rho}{}_{\sigma\mu\nu}
  = \partial_\mu \Gamma^{\rho}{}_{\nu\sigma}
  - \partial_\nu \Gamma^{\rho}{}_{\mu\sigma}
  + \Gamma^{\rho}{}_{\mu\lambda}\Gamma^{\lambda}{}_{\nu\sigma}
  - \Gamma^{\rho}{}_{\nu\lambda}\Gamma^{\lambda}{}_{\mu\sigma}.
  \end{equation}

  \begin{derivationnote}[说明]
  若存在挠率 $T^\lambda{}_{\mu\nu}$,则对易子还会出现 $T$ 的附加项;
  在 Levi--Civita 情形($T=0$)才简化为 \eqref{eq:comm-def}。
  \end{derivationnote}

  \subsubsection{Riemann张量的基本性质}
  用度规降第一指标
  \[
  R_{\rho\sigma\mu\nu} \;\coloneqq\; g_{\rho\lambda}\,R^{\lambda}{}_{\sigma\mu\nu}.
  \]
  则有基本对称性:
  \begin{align}
  &R_{\rho\sigma\mu\nu} = - R_{\sigma\rho\mu\nu},\qquad
  R_{\rho\sigma\mu\nu} = - R_{\rho\sigma\nu\mu},\\
  &R_{\rho\sigma\mu\nu} = R_{\mu\nu\rho\sigma}.
  \end{align}

  \begin{derivationnote}[曲率张量的独立分量个数]
由基本对称性和第一bianchi恒等式可得到 Riemann 张量的独立分量个数为 $\frac{n^2(n^2-1)}{12}$。反对称性告诉我们前后两组坐标,一组的独立个数是$C^2_n = \frac{n(n-1)}{2}$,块对称性,可以把R理解为一个$M \times M$的矩阵,其中$M = \frac{n(n-1)}{2}$,块对称性说明这个矩阵是对称的,因此只有上三角部分的元素是独立的,独立分量个数是$\frac{M(M+1)}{2} = \frac{n^2(n^2-1)}{8}$,再用第一Bianchi恒等式可以把独立分量个数减少到$\frac{n^2(n^2-1)}{12}$。
  \end{derivationnote}
  \paragraph{(1) 第一 Bianchi(代数循环恒等式)}
  \begin{equation}\label{eq:bianchi1}
  R_{\rho[\sigma\mu\nu]} \;=\; 0
  \quad\Longleftrightarrow\quad
  R_{\rho\sigma\mu\nu} + R_{\rho\nu\sigma\mu} + R_{\rho\mu\nu\sigma} = 0.
  \end{equation}

  \paragraph{(2) 第二 Bianchi(微分 Bianchi)}
  \begin{equation}\label{eq:bianchi2}
  \nabla_{[\lambda} R_{\mu\nu]\rho\sigma} \;=\; 0.
  \end{equation}
  \begin{proof}
    给一个简短的说明,曲率张量逐点定义,可以由零联络坐标定理取坐标消去联络。这样协变导数就变成了普通导数,而在这种坐标系下,曲率张量的表达式只含有联络的偏导数项,因此对曲率张量的循环求偏导之和为零。由于该等式在任意坐标系下成立,因此协变形式也成立。
  \end{proof}

  \paragraph{(3) 爱因斯坦张量的协变散度为零}
  将上面的比安基恒等式中的 $\rho$ 和 $\sigma$ 缩并可得:

\[
D_\rho R^\rho_{\lambda\mu\nu} - D_\mu R_{\lambda\nu} + D_\nu R_{\lambda\mu} = 0
\]

再利用度规张量将 $\lambda$ 和 $\nu$ 缩并可得:

\[
D_\rho R^\rho_\mu - D_\mu R + D_\lambda R^\lambda_\mu = 0
\]

整理上式可得(将哑指标 $\lambda$ 和 $\nu$ 统一换为 $\rho$):

\[
D_\rho R^\rho_\mu - D_\mu R + D_\rho R^\rho_\mu = 2D_\rho R^\rho_\mu - D_\mu R = 0
\]

即:

\[
D_\rho \left(R^\rho_\mu - \frac{1}{2}\delta^\rho_\mu R\right) = 0
\]

这正是爱因斯坦张量 $G^\rho_\mu = R^\rho_\mu - \frac{1}{2}\delta^\rho_\mu R$ 的协变散度恒为零:

\[
D_\rho G^\rho_\mu = 0
\]

这正是爱因斯坦在构造引力场方程时需要的性质。

  \subsection{微分同胚映射与李导数}

设 $f: M \rightarrow M'$ 为微分同胚(diffeomorphism),若 $f$ 是一一对应且 $f$ 与 $f^{-1}$ 均为 $C^\infty$ 映射,则称 $M$ 与 $M'$ 互为微分同胚。

在流形上,若一族这样的微分同胚依参数连续变化,可视为描述点沿着流形“流动”的\emph{流形上的流}。由此,微分同胚可以理解为流形自映射的一类光滑变换。

\vspace{0.5em}
考虑从流形 $M$ 到自身的微分同胚映射(或称变换)。由于我们关注的是无穷小的连续变换,因此可以选取流形上一点 $P$ 及其变换后的点 $Q$,在相同的坐标片中分别记为 $x^\mu$ 与 $x'^{\mu}$。  
此时变换的一般形式写作:
\begin{align}
x'^{\mu} = x'^{\mu}(x) \nonumber
\end{align}

\paragraph{两种等价的理解。}
对于这样的微分同胚变换,可以从两种角度理解:
\begin{enumerate}[label=(\arabic*)]
  \item 主动观点:坐标系固定,点随变换而移动;
  \item 被动观点:点保持不动,而坐标系被重新标定。
\end{enumerate}
两者在无穷小时刻等价。

\vspace{0.5em}
采用被动观点时,无穷小的微分同胚变换可表示为:
\begin{align}
x'^{\mu} = x^{\mu} + \epsilon\,\xi^{\mu}(x) \nonumber
\end{align}
其中 $\epsilon$ 为无穷小参数,$\xi^{\mu}(x)$ 为定义在流形上的 $C^\infty$ 矢量场。  
该矢量场描述了点在变换下的平移方向与幅度。

\vspace{0.5em}
由此,流形上所有此类无穷小变换(即 $\xi$ 的集合)在数学上构成一个\textbf{李群}(Lie 群)的无穷小生成元,而这些生成元的集合构成相应的\textbf{李代数}。  
在微分几何的语境下,称 $\xi$ 为流形上微分同胚群的\emph{生成矢量场}。

\begin{definition}[流与生成矢量场]
在流形 $M$ 上取一族微分同胚 $\{\phi_\epsilon\}$,满足 $\phi_0=\mathrm{id}$。在局部坐标 $x^\mu$ 中,其无穷小作用可写成
\begin{align}
x'^{\mu}=\phi_\epsilon^\mu(x)=x^\mu+\epsilon\,\xi^\mu(x)+\mathcal{O}(\epsilon^2) \nonumber
\end{align}
其中 $\xi^\mu(x)$ 为任意 $C^\infty$ 矢量场,称为该流的生成矢量场。把点 $P$ 送到 $Q=\phi_\epsilon(P)$ 的同时,也可视为一次被动坐标变换;在 $\epsilon\Rightarrow0$ 的极限下两种观点等价。
\end{definition}

\begin{derivationnote}[符号说明]
\begin{itemize}
  \item $\phi_\epsilon: M \to M$ 表示以参数 $\epsilon$ 标记的一族光滑微分同胚;当 $\epsilon=0$ 时为恒等映射。
  \item $x^\mu$ 与 $x'^{\mu}$ 分别表示点 $P$ 与其变换后点 $Q=\phi_\epsilon(P)$ 在同一坐标片中的坐标分量。
  \item $\xi^\mu(x)$ 是定义在 $M$ 上的平滑矢量场,给出点在变换下的无穷小位移方向与速率。
  \item $\epsilon$ 为实数参数,描述变换的“强度”;取 $\epsilon\to 0$ 时,变换趋于恒等映射。
\end{itemize}
\end{derivationnote}

为比较同一点的张量值,引入记号 $T(P\Rightarrow Q)$:先在变换下把 $T$ 拖到 $Q$,再把其值回比较到 $P$。沿矢量场 $\xi$ 的\emph{李导数}定义为
\begin{align}
\mathcal{L}_\xi T \equiv \lim_{\epsilon\to 0}\frac{T(Q)-T(P\Rightarrow Q)}{\epsilon} \nonumber
\end{align}
它描述了张量场在 $\xi$ 生成的流下的一阶变化率。

\paragraph{标量场:}
对标量 $f$,有 $F(P\Rightarrow Q)=F(P)=f(P)$,因此
\begin{align}
\mathcal{L}_\xi f=\xi^\mu\partial_\mu f . \nonumber
\end{align}

\paragraph{逆变矢量(切矢):}
取曲线 $x^\mu(\lambda)$ 的切矢 $k^\mu=\frac{dx^\mu}{d\lambda}$。在
$x^\mu\mapsto x^\mu+\epsilon\,\xi^\mu(x)$ 下,
\begin{align}
k^\mu(P\Rightarrow Q)
&=\frac{d}{d\lambda}\!\left[x^\mu+\epsilon\,\xi^\mu(x)\right]_{P} \nonumber\\
&= k^\mu(P)+\epsilon\,k^\nu\partial_\nu\xi^\mu\big|_{P}. \nonumber
\end{align}
将 $k^\mu(Q)$ 在 $P$ 处展开:
\begin{align}
k^\mu(Q)=k^\mu(P)+\epsilon\,\xi^\nu\partial_\nu k^\mu\big|_{P}. \nonumber
\end{align}
于是
\begin{align}
(\mathcal{L}_\xi k)^\mu
=\lim_{\epsilon\to 0}\frac{k^\mu(Q)-k^\mu(P\Rightarrow Q)}{\epsilon}
=\xi^\nu\partial_\nu k^\mu-k^\nu\partial_\nu\xi^\mu .
\tag{1}
\end{align}

\paragraph{协变矢量:}由 Leibniz 相容性(见下)与
$\mathcal{L}_\xi(p_\mu k^\mu)=\xi^\nu\partial_\nu(p_\mu k^\mu)$,结合式(1) 得
\begin{align}
(\mathcal{L}_\xi p)_\mu=\xi^\nu\partial_\nu p_\mu+p_\nu\,\partial_\mu\xi^\nu .
\tag{2}
\end{align}

\begin{derivationnote}[一般张量的李导数]
李导数与张量运算(张量积、缩并、指标置换)相容,并在标量上退化为方向导数。
据此从式(1)(2) 推广到任意 $(r,s)$ 型张量 $T$:
\begin{align}
(\mathcal{L}_\xi T)^{\alpha_1\cdots\alpha_r}{}_{\beta_1\cdots\beta_s}
&= \xi^\rho\partial_\rho T^{\alpha_1\cdots\alpha_r}{}_{\beta_1\cdots\beta_s} \nonumber\\
&\quad-\sum_{i=1}^{r} T^{\alpha_1\cdots\rho\cdots\alpha_r}{}_{\beta_1\cdots\beta_s}\,\partial_\rho\xi^{\alpha_i} \nonumber\\
&\quad+\sum_{j=1}^{s} T^{\alpha_1\cdots\alpha_r}{}_{\beta_1\cdots\rho\cdots\beta_s}\,\partial_{\beta_j}\xi^{\rho}.
\tag{3}
\end{align}
\end{derivationnote}

\paragraph{要点与常用形式。}
\begin{itemize}
  \item 李导数不仅与张量本身有关,也取决于生成矢量场 $\xi(x)$。
  \item 该定义对任意可微流形都成立,不依赖度规或联络。
  \item 在需要时,可把分量公式中的普通导数以协变导数替换并自动补偿联络项;典型计算如
  \begin{align}
  (\mathcal{L}_\xi g)_{\mu\nu}
  &= \xi^\rho\partial_\rho g_{\mu\nu}
  + g_{\rho\nu}\partial_\mu\xi^\rho
  + g_{\mu\rho}\partial_\nu\xi^\rho \nonumber\\
  &= \nabla_\mu\xi_\nu+\nabla_\nu\xi_\mu . \label{eq:killing}
  \end{align}
\end{itemize}
\begin{figure}[htbp]
  \centering
  \includegraphics[width=0.6\textwidth]{figs/M.png}
  \caption{主要数学结构及其关系示意图。}
\end{figure}

\subsection{等度规映射与Killing矢量场}

\begin{definition}[等度规映射]
对于黎曼空间$(M,g_{\mu\nu})$,保持度规张量不变的微分同胚变换称为等度规映射(isometry)。由于微分同胚变化与张量运算的相容性,可得:
\begin{align}
g_{\mu\nu}(P \Rightarrow Q) = g_{\mu\nu}(Q)
\end{align}
\end{definition}
这其实就是说度规的李导数为0,根据\ref{eq:killing}可得$Killing$方程:
\n{Killing方程是超定的,方程给出的约束条件个数大于未知函数个数,因此一般解不存在,只有在高度对称的空间中才可能存在非平凡解。}
\begin{theorem}[Killing方程]
等度规映射的无穷小形式由Killing方程描述:
\begin{align}
\mathcal{L}_\xi g_{\mu\nu} = \nabla_\mu\xi_\nu+\nabla_\nu\xi_\mu = 0
\end{align}
满足$Killing$方程的微分矢量场$\xi^\mu(x)$称为Killing矢量场,是等度规变换的生成元。
\end{theorem}

\begin{example}[欧式平面的Killing矢量场]
对于欧式平面度规$g_{ij} = \delta_{ij}$,求解Killing方程可得3个线性独立的Killing矢量场:
\begin{align}
\xi_{(1)}^\mu &= (1,0) \quad \text{(x方向平移)} \\
\xi_{(2)}^\mu &= (0,1) \quad \text{(y方向平移)} \\
\xi_{(3)}^\mu &= (-y,x) \quad \text{(绕原点转动)}
\end{align}
这对应于欧式平面的3个独立几何对称性。
\end{example}

\begin{example}[闵可夫斯基时空的Killing矢量场]
对于闵氏度规$g_{\mu\nu} = \eta_{\mu\nu}$,存在10个线性独立的Killing矢量场,分别对应于:
\begin{itemize}
\item 4个平移变换(时空平移对称性)
\item 6个Lorentz变换(3个空间转动和3个boost变换)
\end{itemize}
时间平移对应的Killing矢量为:
\begin{align}
\xi^\mu = (1,0,0,0)
\end{align}
\end{example}

\begin{definition}[稳定时空]
如果时空区域存在类时Killing矢量场$\xi^\mu$,即满足:
\begin{align}
g_{\mu\nu}\xi^\mu\xi^\nu < 0
\end{align}
则称该时空区域是稳定的(stationary)。
\end{definition}

\begin{theorem}[适配坐标系与时间平移不变性]
对于类时Killing矢量场$\xi^\mu$,可以构造适配坐标系,使得:
\begin{align}
\xi^\mu(x) = \frac{dx^\mu}{d\lambda}
\end{align}
在适配坐标系下,度规不显含时间坐标:
\begin{align}
\partial_0 g_{\mu\nu} = 0
\end{align}
从而时间平移变换$x^0 \to x^0 + b$是等度规变换。
\end{theorem}

\begin{derivationnote}[适配坐标系的构造细节]
\end{derivationnote}



\begin{definition}[循环坐标与守恒量]
不显含于度规场中的坐标称为循环坐标,从而度规对这个坐标的偏导数为0对应一种守恒量。
\end{definition}
时间平移不变性(由类时Killing矢量场描述)将导致能量守恒,这是广义相对论框架下的诺特定理的具体体现。
\begin{definition}[最大对称空间]
具有最大数目独立Killing矢量场的空间称为最大对称空间。n维黎曼空间最多有$\frac{n(n+1)}{2}$个独立Killing矢量场。
\end{definition}


\subsection{黎曼空间中的常用计算}

在黎曼几何中,度规张量$g_{\mu\nu}$的行列式通常简记为$g$,即:
\begin{align}
g \equiv \det(g_{\mu\nu})
\end{align}
由于在物理应用中度规通常具有负号约定,实际计算中常用的是$\-g$。


\n{度规行列式$g$包含了度规张量的全部信息,在坐标变换下按特定规律变换。}

\begin{theorem}[度规行列式的导数]
度规行列式$g$对度规分量的偏导数为:
\begin{align}
\frac{\partial g}{\partial g_{\mu\nu}} = \Delta^{\mu\nu} = g g^{\mu\nu}
\end{align}
其中$\Delta^{\mu\nu}$是$g_{\mu\nu}$的代数余子式。由此可得行列式对坐标的偏导数:
\begin{align}
\partial_\rho g = g g^{\mu\nu} \partial_\rho g_{\mu\nu} = -g g_{\mu\nu} \partial_\rho g^{\mu\nu}
\end{align}
\end{theorem}
\n{一般度规的行列式是负的,因此$-g$是正值。}
\begin{theorem}[Christoffel联络的缩并]
Christoffel联络$\Gamma^\mu_{\rho\mu}$的缩并形式为:
\begin{align}
\Gamma^\mu_{\rho\mu} = \frac{1}{2} g^{\mu\nu} \partial_\rho g_{\mu\nu} = \frac{1}{2g} \partial_\rho g = \partial_\rho \ln\sqrt{-g} = \frac{1}{\sqrt{-g}} \partial_\rho \sqrt{-g}
\end{align}
这个结果在协变散度的计算中具有核心重要性。
\end{theorem}

\begin{definition}[协变散度]
对于任意张量场,其协变散度定义为该张量的协变导数在某个指标上的缩并。以逆变矢量场$J^\mu(x)$为例,其协变散度为:
\begin{align}
\nabla_\rho J^\rho = \partial_\rho J^\rho + \Gamma^\mu{}_{\rho\mu} J^\rho
\end{align}
\end{definition}

\begin{theorem}[矢量场的协变散度简化]
利用Christoffel联络的缩并关系,逆变矢量场的协变散度可简化为:
\begin{align}
\nabla_\rho J^\rho = \frac{1}{\sqrt{-g}} \partial_\rho (\sqrt{-g} J^\rho)
\end{align}
当$\nabla_\rho J^\rho = 0$时,称$J^\mu$为守恒流。
\end{theorem}

\n{这个简化形式极大方便了实际计算,将复杂的张量运算转化为普通导数运算。}

在广义相对论中,能量-动量张量的协变散度为零对应着局域能量-动量守恒定律。对于电磁场,四电流密度的协变散度为零对应着电荷守恒。

\begin{theorem}[二阶张量场的协变散度]
对于二阶逆变张量场$T^{\mu\nu}(x)$,其协变散度为:
\begin{align}
\nabla_\rho T^{\rho\nu} = \frac{1}{\sqrt{-g}} \partial_\rho (\sqrt{-g} T^{\rho\nu}) + \Gamma^\nu{}_{\mu\rho} T^{\rho\mu}
\end{align}
特别地,当$T^{\mu\nu}$是反对称张量时,第二项为零,简化为:
\begin{align}
\nabla_\rho T^{\rho\nu} = \frac{1}{\sqrt{-g}} \partial_\rho (\sqrt{-g} T^{\rho\nu})
\end{align}
\end{theorem}

\begin{theorem}[全反称张量场的协变散度]
对于任意阶的全反称逆变张量场$H^{\mu\nu\cdots\sigma}(x)$,其协变散度满足:
\begin{align}
\nabla_\rho H^{\rho\nu\cdots\sigma} = \frac{1}{\sqrt{-g}} \partial_\rho (\sqrt{-g} H^{\rho\nu\cdots\sigma})
\end{align}
这一结果可以视为前述定理在高阶反对称张量情形的推广。
\end{theorem}


