\section{爱因斯坦场方程和广义相对论的实验检验}

\subsection{引力几何化与弱场近似}
\label{subsec:grav_geom_weak_field}

等效原理告诉我们,引力和惯性里局域不可区分,所以我们可以把引力效应等价于坐标变换的效应,最终归纳于时空的几何效应。考虑自由粒子在惯性系中的运动方程:
\begin{align}
\frac{d^2 X^\mu}{d\tau^2} = 0,
\end{align}
其中 \(X^\mu\) 是闵可夫斯基坐标。通过一般坐标变换 \(x^\mu = x^\mu(X^\nu)\) 引入非惯性系后,运动方程变为
\begin{align}
\frac{d^2 x^\mu}{d\tau^2} = - \Gamma^\mu_{\nu\lambda} \frac{dx^\nu}{d\tau} \frac{dx^\lambda}{d\tau}
\end{align}
把这个方程与牛顿方程对比,可得联络在此处的作用类似于惯性力的场强,由等效原理,引力效应也可以通过联络来描述,联络蕴含了时空的几何结构,因此这种思想就是引力的几何化。粒子物理无论是在惯性力场还是引力场中,其运动方程都是测地线方程,力的效应在联络中称为时空本身的性质。
联络是引力场强,度规的导数,对比电势和电场,度规就是相对论框架下的引力势。
\subsubsection*{弱场近似下的运动方程}

在弱场近似中,引入以下条件:
\begin{enumerate}
\item 引力场弱:\(g_{\mu\nu} = \eta_{\mu\nu} + h_{\mu\nu}\),且 \(|h_{\mu\nu}| \ll 1\)。
\item 引力场稳态:\(h_{\mu\nu,0} = 0\)。
\item 引力场空间缓变:\(|h_{\mu\nu,i}| \ll 1\),其中拉丁指标 \(i,j,k\) 表示空间坐标。
\item 粒子低速运动:\(\left| \frac{dx^i}{dx^0} \right| \ll 1\)。
\end{enumerate}

在这些条件下,联络可近似为:
\begin{align}
\Gamma^\mu{}_{\nu\lambda} = \frac{1}{2} \eta^{\mu\rho} (\partial_\lambda h_{\rho\nu} + \partial_\nu h_{\rho\lambda} - \partial_\rho h_{\nu\lambda}).
\end{align}
测地线方程可简化为:
\begin{align}
\frac{d^2 x^0}{d\tau^2} &= 0, \\
\frac{d^2 x^i}{d\tau^2} &= -\Gamma^i{}_{00} \left( \frac{dx^0}{d\tau} \right)^2.
\end{align}
利用 \(x^0 = t\)(时间坐标)和低速条件,得到:
\begin{align}
\frac{d^2 x^i}{dt^2} = -\partial_i \varphi,
\end{align}
其中牛顿引力势 \(\varphi\) 与度规扰动的关系为(自然单位制):
\begin{align}
\varphi = -\frac{1}{2} h_{00}, \quad g_{00} = -1 - 2\varphi = -(1-\frac{2GM}{r})
\end{align}
还原光速,$g_00$等于0时,$R_g = \frac{2GM}{c^2}$称为施瓦西半径,当距离远大于施瓦西半径时,度规退化为Minkovski度规。这表明在弱场低速近似下,广义相对论还原为牛顿引力理论。

\begin{example}[太阳的引力半径]
对于质量为 \(M\) 的球对称引力源,牛顿引力势为 \(\varphi = -GM/r\)。弱场条件要求 \(r \gg R_g\),其中引力半径(史瓦西半径)为:
\begin{align}
R_g = \frac{2GM}{c^2}.
\end{align}
太阳的质量 \(M \approx 2 \times 10^{33} \, \text{g}\),引力半径 \(R_g \approx 3 \, \text{km}\),而实际半径约 \(7 \times 10^5 \, \text{km}\),故弱场条件高度满足。
\end{example}

\subsubsection*{爱因斯坦场方程及其弱场极限}

\begin{theorem}[爱因斯坦场方程]
引力场方程的最一般形式为:
\begin{align}
G_{\mu\nu} + \Lambda g_{\mu\nu} = \kappa T_{\mu\nu},
\end{align}
其中 \(G_{\mu\nu} = R_{\mu\nu} - \frac{1}{2} g_{\mu\nu} R\) 是爱因斯坦张量,\(\Lambda\) 是宇宙学常数,\(\kappa\) 是相对论引力常数,\(T_{\mu\nu}\) 是能量动量张量。当 \(\Lambda = 0\) 时,方程为:
\begin{align}
G_{\mu\nu} = \kappa T_{\mu\nu}.
\end{align}
在真空区域(\(T_{\mu\nu} = 0\)),场方程简化为 \(R_{\mu\nu} = 0\),对应 Ricci 平坦时空。
\end{theorem}

流体四速度满足归一化条件$g_{\mu\nu} u^\mu u^\nu = -1$。
静止流体三矢量均为0,可得$g_{00}u^0u^0 = -1$。
可得其四速度为:
\begin{align}
u^\mu = (-g_{00})^{-1/2} (1, 0, 0, 0).
\end{align}
由此可得对应的能量动量张量,其非零分量为:
\begin{align}
T^{00} = \rho (-g_{00})^{-1}, \quad T = g_{\mu\nu} T^{\mu\nu} = -\rho.
\end{align}
在弱场近似下,爱因斯坦场方程的 00 分量化为:
\begin{align}
    R^{00} = \kappa \left( T^{00} - \frac{1}{2} g^{00} T \right)
\end{align}
带入可得,
\begin{align}
\nabla^2 \varphi = \frac{1}{2} \kappa \rho.
\end{align}
与牛顿引力场方程 \(\nabla^2 \varphi = 4\pi G \rho\) 对比,可得:
\begin{align}
\kappa = 8\pi G.
\end{align}
恢复光速后,\(\kappa = 8\pi G / c^4\).

\begin{derivationnote}[宇宙学常数的影响]
若保留宇宙学常数项,弱场近似下牛顿方程修正为:
\begin{align}
\nabla^2 \varphi = 4\pi G \rho - \Lambda.
\end{align}
观测表明 \(\Lambda\) 极小,在太阳系尺度可忽略。
\end{derivationnote}
