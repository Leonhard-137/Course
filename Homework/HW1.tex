% !TeX program = xelatex
% !TeX encoding = UTF-8

\documentclass[11pt,a4paper]{ctexart}

\usepackage{geometry}
\geometry{a4paper,margin=2.5cm,includeheadfoot}

\usepackage{amsmath,amssymb,bm}
\usepackage{xcolor}  % 添加 xcolor 包
\usepackage{hyperref}
\hypersetup{colorlinks=true,linkcolor=blue!60!black,urlcolor=blue!60!black,citecolor=blue!60!black}

\setlength{\parindent}{2em}
\setlength{\parskip}{0.5em}

% ---- 题目/解答环境 ----
\newcounter{problem}
\newenvironment{problem}[1][]{
  \refstepcounter{problem}% 
  \noindent\textbf{题目 \theproblem}\if\relax\detokenize{#1}\relax\else\ \textbf{(#1)}\fi\quad
}{\par}

\newenvironment{solution}{
  \noindent\textbf{解答}\quad
}{\par}

% ---- 页眉页脚 ----
\usepackage{fancyhdr}
\pagestyle{fancy}
\setlength{\headheight}{14.5pt}
\fancyhf{}
\fancyhead[L]{广义相对论 作业}
\fancyhead[R]{肖志坤}
\fancyfoot[C]{\thepage}

\title{\vspace{-1.5em}\textbf{广义相对论:第一次作业}}
\author{肖志坤}
\date{\today}

\begin{document}
\maketitle
\thispagestyle{fancy}

% ===== 第 1 题 =====
\begin{problem}
对于电磁场,我们已知其动量-能量-应力张量
\begin{align}
T^{\mu\nu}
&=\frac{1}{\mu_0}\Big(F^{\nu\sigma}F_{\sigma}{}^{\mu}-\tfrac14\,\eta^{\mu\nu}F_{\rho\sigma}F^{\rho\sigma}\Big)
\end{align}
证明它是无迹的,即 $\eta^{\mu\nu}T_{\mu\nu}=0$。
\end{problem}

\begin{solution}
将 $T^{\mu\nu}$ 代入得
\begin{align}
\eta_{\mu\nu}T^{\mu\nu}
&=\frac{1}{\mu_0}\left(\eta_{\mu\nu}F^{\nu\sigma}F_{\sigma}{}^{\mu}
-\frac14\,\eta_{\mu\nu}\eta^{\mu\nu}F_{\rho\sigma}F^{\rho\sigma}\right) \nonumber \\
&=\frac{1}{\mu_0}\left(F^{\sigma\nu}F_{\sigma\nu}
-\frac14\times4\,F_{\rho\sigma}F^{\rho\sigma}\right) = 0
\end{align}
其中 $\mathrm{tr}(\eta^{\mu\nu})=4$,因此 $T^\mu{}_\mu=0$。
\end{solution}

% ===== 第 2 题 =====
\begin{problem}
利用电磁场的能量-动量-应力张量 $T^{\mu\nu}$ 和 Maxwell 方程,证明:
\begin{align}
\partial_{\mu}T^{\mu\nu} &= -F^{\nu}{}_{\sigma}J^{\sigma}
\label{equ:EMconst}
\end{align}
实际上本题反映了电磁场的能动量守恒,请简要说明。
\end{problem}

\begin{solution}
张量形式的 Maxwell 方程为:
\begin{align}
\partial_{\mu}F^{\nu\mu} &= \mu_0 J^{\nu} \\
\partial_{\mu}F_{\nu\sigma} + \partial_{\nu}F_{\sigma\mu} + \partial_{\sigma}F_{\mu\nu} &= 0
\end{align}
先对能动张量一部分求微分:
\begin{align}
\partial_{\mu}F^{\mu}{}_{\sigma}F^{\nu\sigma}
&= \partial^{\mu}F_{\mu \sigma}F^{\nu \sigma} \nonumber \\
&= F_{\mu \sigma}\partial^{\mu}F^{\nu \sigma} +  F^{\nu \sigma}\partial^{\mu}F_{\mu \sigma} \nonumber \\
&= F_{\mu \sigma}\partial^{\mu}F^{\nu \sigma} -  F^{\nu \sigma} \mu_0 J_{\sigma}
\end{align}
这里第二项除以 $\mu_0$ 就等于须证式 \eqref{equ:EMconst} 的右边,只需证剩下部分为 0。略去 $\mu_0$, 其余部分的和为:
\begin{align}
F_{\mu \sigma}\partial^{\mu}F^{\nu \sigma}
- \frac{1}{4}\eta^{\mu \nu}\partial_{\mu}\left(F_{\rho \sigma}F^{\rho \sigma}\right)
&= F_{\mu \sigma}\partial^{\mu}F^{\nu\sigma}
- \frac{1}{2} F_{\rho \sigma}\partial^{\nu}F^{\rho \sigma} \nonumber \\
&= \frac{1}{2}\left(F_{\mu \sigma}\partial^{\mu}F^{\nu\sigma} + F_{\mu \sigma}\partial^{\nu}F^{\nu\sigma} - F_{\mu \sigma}\partial^{\nu}F^{\mu \sigma}\right) \nonumber \\
&= \frac{1}{2}\left(F_{\sigma \mu}\partial^{\sigma}F^{\nu \mu} - F_{\mu \sigma}\partial^{\mu}F^{\sigma \nu}- F_{\mu \sigma}\partial^{\nu}F^{\mu \sigma}\right) \nonumber \\
&= -\frac{1}{2}F_{\mu\sigma}\left(\partial^{\sigma}F^{\nu \mu} + \partial^{\mu}F^{\sigma \nu} + \partial^{\nu}F^{\mu \sigma}\right) \nonumber \\
&= 0
\end{align}
其中,第三行第一项交换了哑指标字母,最后一步利用了 Maxwell 方程的无源方程部分。因此式 \eqref{equ:EMconst} 得证。
\begin{align}
\partial_{\mu}T^{\mu\nu} &= -F^{\nu}{}_{\sigma}J^{\sigma}
\end{align}
实际上是能动量守恒的一个具体形式,取$\nu = 0$和$\nu  = 1,2,3$分别展开,这个方程的矢量形式展开后是:
\begin{align}
\frac{\partial T^{00}}{\partial t} + \nabla \cdot \mathbf{S} &= -\mathbf{E} \cdot \mathbf{J} \\
\frac{\partial \mathbf{P}}{\partial t} + \nabla \cdot \mathbf{T} &= \mathbf{F}
\end{align}
其中,$T^{00}$ 表示电磁场的能量密度,$\mathbf{S}$ 是 Poynting 向量,其中,$\mathbf{P}$ 是电磁场的动量密度,$\mathbf{T}$ 是电磁场的动量流密度,$\mathbf{F}$ 是电磁力密度。
从这个第一个可以看出,电磁场的能量密度会随着Poynting矢量在空间中传播,也会因为与物质的相互作用而产生能量的变化,从而满足整体的能量守恒。
第二个方程与能量守恒方程类似,电磁场动量由于动量流的散度以及电磁力对物质的作用会发生变化,从而满足整体的动量守恒。 
\end{solution}


\begin{problem}
\textbf{商定理}:设在某坐标系中有张量关系
\begin{align}
A^\mu(x) &= B^\mu{}_{\alpha}(x)\,C^{\alpha}(x),
\end{align}
其中 $A^\mu$ 为 $(1,0)$ 张量、$B^\mu{}_{\alpha}$ 为 $(1,1)$ 张量。
证明 $C^{\alpha}$ 为 $(1,0)$ 张量,即在坐标变换 $x\mapsto x'$ 下满足
\begin{align}
C'^{\,\alpha}(x') &= \frac{\partial x'^{\alpha}}{\partial x^\beta}C^\beta(x)
\end{align}
\end{problem}

\begin{solution}
张量分量在坐标变换下的变换律为
\begin{align}
A'^{\mu}(x') &= \frac{\partial x'^{\mu}}{\partial x^\nu}\,A^\nu(x), \\
B'^{\mu}{}_{\alpha}(x') &= \frac{\partial x'^{\mu}}{\partial x^\nu}\,
                         \frac{\partial x^\beta}{\partial x'^{\alpha}}\,
                         B^{\nu}{}_{\beta}(x)
\end{align}
由给定关系 $A^\mu = B^\mu{}_{\alpha}C^{\alpha}$ 得
\begin{align}
A'^{\mu}(x')
  &= \frac{\partial x'^{\mu}}{\partial x^\nu}\,B^\nu{}_{\beta}(x)\,C^\beta(x) \nonumber \\
  &= \Bigg(\frac{\partial x'^{\mu}}{\partial x^\nu}
           \frac{\partial x^\beta}{\partial x'^{\alpha}}
           B^\nu{}_{\beta}(x)\Bigg)
     \Bigg(\frac{\partial x'^{\alpha}}{\partial x^\gamma}C^\gamma(x)\Bigg) \nonumber \\
  &= B'^{\mu}{}_{\alpha}(x')\;
    \Bigg(\frac{\partial x'^{\alpha}}{\partial x^\gamma}C^\gamma(x)\Bigg)
\end{align}
而在 $x'$ 系同样有 $A'^{\mu} = B'^{\mu}{}_{\alpha}C'^{\,\alpha}$。若 $B^\mu{}_{\alpha}$在该点可逆,
则可得
\begin{align}
C'^{\,\alpha}(x') = \frac{\partial x'^{\alpha}}{\partial x^\gamma}C^\gamma(x)
\end{align}
\end{solution}

\begin{problem}
设 $T^{\mu_1\cdots\mu_p}{}_{\nu_1\cdots\nu_q}(x)$ 为 $(p,q)$ 张量。
证明:将其中一上指标与一下指标缩并得到的
\begin{align}
S^{\mu_1\cdots\widehat{\mu_r}\cdots\mu_p}{}_{\nu_1\cdots\widehat{\nu_s}\cdots\nu_q}(x)
&:= T^{\mu_1\cdots\alpha\cdots\mu_p}{}_{\nu_1\cdots\alpha\cdots\nu_q}(x)
\end{align}
是一个 $(p-1,q-1)$ 张量。
\end{problem}

\begin{solution}
张量分量在坐标变换 $x\mapsto x'$ 下满足
\begin{align}
T'^{\mu_1\cdots\mu_p}{}_{\nu_1\cdots\nu_q}(x')
&= \frac{\partial x'^{\mu_1}}{\partial x^{\alpha_1}}\cdots
  \frac{\partial x'^{\mu_p}}{\partial x^{\alpha_p}}\,
  \frac{\partial x^{\beta_1}}{\partial x'^{\nu_1}}\cdots
  \frac{\partial x^{\beta_q}}{\partial x'^{\nu_q}}\;
  T^{\alpha_1\cdots\alpha_p}{}_{\beta_1\cdots\beta_q}(x)
\end{align}
把第 $r$ 个上指标与第 $s$ 个下指标缩并,有
\begin{align}
S'^{\mu_1\cdots\widehat{\mu_r}\cdots\mu_p}{}_{\nu_1\cdots\widehat{\nu_s}\cdots\nu_q}(x')
&= T'^{\mu_1\cdots\alpha\cdots\mu_p}{}_{\nu_1\cdots\alpha\cdots\nu_q}(x') \nonumber \\
&= \Bigg(\prod_{\substack{i=1\\ i\neq r}}^{p}\frac{\partial x'^{\mu_i}}{\partial x^{\alpha_i}}\Bigg)
   \frac{\partial x'^{\alpha}}{\partial x^{\rho}}\;
  \Bigg(\prod_{\substack{j=1\\ j\neq s}}^{q}\frac{\partial x^{\beta_j}}{\partial x'^{\nu_j}}\Bigg)
   \frac{\partial x^{\sigma}}{\partial x'^{\alpha}}\;
  T^{\alpha_1\cdots\rho\cdots\alpha_p}{}_{\beta_1\cdots\sigma\cdots\beta_q}(x)
\end{align}
利用链式法则得到 Kronecker 符号
\begin{align}
\frac{\partial x'^{\alpha}}{\partial x^{\rho}}\,
\frac{\partial x^{\sigma}}{\partial x'^{\alpha}} = \delta^{\sigma}_{\rho},
\end{align}
于是
\begin{align}
S'^{\mu_1\cdots\widehat{\mu_r}\cdots\mu_p}{}_{\nu_1\cdots\widehat{\nu_s}\cdots\nu_q}(x')
&= \Bigg(\prod_{\substack{i=1\\ i\neq r}}^{p}\frac{\partial x'^{\mu_i}}{\partial x^{\alpha_i}}\Bigg)
  \Bigg(\prod_{\substack{j=1\\ j\neq s}}^{q}\frac{\partial x^{\beta_j}}{\partial x'^{\nu_j}}\Bigg)
  T^{\alpha_1\cdots\lambda\cdots\alpha_p}{}_{\beta_1\cdots\lambda\cdots\beta_q}(x)
\end{align}
其中把哑指标 $\rho,\sigma$ 重命名为 $\lambda$。这正是 $(p-1,q-1)$ 张量的变换律,故缩并后仍为张量。
\end{solution}

\begin{problem}
证明:逆变矢量的平行输运在坐标变换下仍保持逆变矢量的性质。即若沿位移
$dx^\nu$ 从点 $p$ 到 $q=p+dx$ 的一阶平行输运写为
\begin{align}
B^\gamma(p\to q) &= B^\gamma(p)-\Gamma^\gamma_{\mu\nu}(p)\,B^\mu(p)\,dx^\nu,
\end{align}
则在任意坐标变换 $x\mapsto x'$ 下有
\begin{align}
B'^{\gamma}(p\to q) &= \left[\frac{\partial x'^{\gamma}}{\partial x^{\alpha}}\right]_{q}\,
      B^{\alpha}(p\to q)
\end{align}
\end{problem}

\begin{solution}
在 $x$ 系:\quad
$B^\alpha(q)=B^\alpha(p)-\Gamma^\alpha{}_{\beta\delta}(p)\,B^\beta(p)\,dx^\delta$。

在 $x'$ 系按同一定义:
\begin{align}
B'^{\gamma}(p\to q) &= B'^{\gamma}(p)-\Gamma'^{\gamma}{}_{\mu\nu}(p)\,B'^{\mu}(p)\,dx'^{\nu}
\end{align}
代入变换律并约去中间雅可比(此处量均在 $p$ 取值):
\begin{align*}
B'^{\gamma}(p\to q)
&=\frac{\partial x'^{\gamma}}{\partial x^{\alpha}}\,B^{\alpha}
 -\frac{\partial x'^{\gamma}}{\partial x^{\rho}}
  \Gamma^{\rho}{}_{\beta\delta}\,B^{\beta}dx^{\delta}
 -\frac{\partial x'^{\gamma}}{\partial x^{\rho}}
  \frac{\partial^{2}x^{\rho}}{\partial x'^{\mu}\partial x'^{\nu}}
  \frac{\partial x'^{\mu}}{\partial x^{\beta}}
  \frac{\partial x'^{\nu}}{\partial x^{\delta}}
  B^{\beta}dx^{\delta}
\end{align*}
对恒等式 \(\dfrac{\partial x^\rho}{\partial x'^\mu}\dfrac{\partial x'^\mu}{\partial x^\beta}=\delta^\rho{}_\beta\) 对 $x^\delta$ 求导,得
\begin{align}
\frac{\partial x'^{\gamma}}{\partial x^{\rho}}
\frac{\partial^{2}x^{\rho}}{\partial x'^{\mu}\partial x'^{\nu}}
\frac{\partial x'^{\mu}}{\partial x^{\beta}}
\frac{\partial x'^{\nu}}{\partial x^{\delta}} = -\,\frac{\partial^{2}x'^{\gamma}}{\partial x^{\beta}\partial x^{\delta}}
\end{align}
因此
\begin{align}
B'^{\gamma}(p\to q)
&=\frac{\partial x'^{\gamma}}{\partial x^{\alpha}}\,B^{\alpha}
 -\frac{\partial x'^{\gamma}}{\partial x^{\rho}}
  \Gamma^{\rho}{}_{\beta\delta}\,B^{\beta}dx^{\delta}
 +\frac{\partial^{2}x'^{\gamma}}{\partial x^{\beta}\partial x^{\delta}}\,B^{\beta}dx^{\delta}
\end{align}
另一方面,雅可比在 $q=p+dx$ 的一阶展开:
\begin{align}
\Big[\frac{\partial x'^{\gamma}}{\partial x^{\alpha}}\Big]_q
&=\frac{\partial x'^{\gamma}}{\partial x^{\alpha}}
 +\frac{\partial^{2}x'^{\gamma}}{\partial x^{\alpha}\partial x^{\delta}}\,dx^{\delta}
\end{align}
注意到:
\begin{align*}
\Big[\frac{\partial x'^{\gamma}}{\partial x^{\alpha}}\Big]_q
\big(B^{\alpha}-\Gamma^{\alpha}{}_{\beta\delta}B^{\beta}dx^{\delta}\big)
&=\frac{\partial x'^{\gamma}}{\partial x^{\alpha}}\,B^{\alpha}
 -\frac{\partial x'^{\gamma}}{\partial x^{\rho}}
  \Gamma^{\rho}{}_{\beta\delta}\,B^{\beta}dx^{\delta}
 +\frac{\partial^{2}x'^{\gamma}}{\partial x^{\beta}\partial x^{\delta}}\,B^{\beta}dx^{\delta}
\end{align*}
与上式相同,故
\begin{align}
B'^{\gamma}(p\to q)
&= \Big[\frac{\partial x'^{\gamma}}{\partial x^{\alpha}}\Big]_q\,
\big(B^{\alpha}(p)-\Gamma^{\alpha}{}_{\beta\delta}(p)\,B^{\beta}(p)\,dx^{\delta}\big)
=\Big[\frac{\partial x'^{\gamma}}{\partial x^{\alpha}}\Big]_q\,B^{\alpha}(p\to q)
\end{align}
即平移后的 $B$ 在 $q$ 点按逆变矢量变换。
\end{solution}

\begin{problem}
  黎曼几何基本定理:在一个给定的流形上,与流形上某给定度规相关的无挠联络只有一个。
\end{problem}

\begin{solution}
  展开度规的协变导数并轮换其指标
  \begin{align}
    \nabla_{\rho} g_{\mu \nu} &= \partial_{\rho} g_{\mu \nu} - \Gamma_{\mu \rho}^{\lambda} g_{\lambda \nu} - \Gamma_{\nu \rho}^{\lambda} g_{\mu \lambda} = 0, \\
    \nabla_{\mu} g_{\nu \rho} &= \partial_{\mu} g_{\nu \rho} - \Gamma_{\mu \nu}^{\lambda} g_{\lambda \rho} - \Gamma_{\nu \mu}^{\lambda} g_{\lambda \rho} = 0, \\
    \nabla_{\nu} g_{\mu \rho} &= \partial_{\nu} g_{\mu \rho} - \Gamma_{\nu \mu}^{\lambda} g_{\lambda \rho} - \Gamma_{\mu \nu}^{\lambda} g_{\lambda \rho} = 0
  \end{align}

  \text{第一式减去第二式和第三式的和:}
  \begin{align}
    \partial_{\rho} g_{\mu \nu} - \partial_{\mu} g_{\nu \rho} - \partial_{\nu} g_{\mu \rho} + 2 \Gamma_{\mu \nu}^{\lambda} g_{\lambda \rho} &= 0
  \end{align}

  \text{双边乘以度规张量$g^{\rho \sigma}$,可以得到:}
  \begin{align}
    \Gamma_{\mu \nu}^{\sigma} &= \frac{1}{2} g^{\rho \sigma} \left( \partial_{\mu} g_{\nu \rho} + \partial_{\nu} g_{\mu \rho} - \partial_{\rho} g_{\mu \nu} \right)
  \end{align}
\end{solution}
\begin{problem}
第一题:曲线的弧长可以表示为:

$$ S=\int_{\lambda_{i}}^{\lambda_{f}} ds=\int_{\lambda_{i}}^{\lambda_{f}}\sqrt{g_{\mu\nu}\frac{dx^{\mu}}{d\lambda}\frac{dx^{\nu}}{d\lambda}}d\lambda\qquad(1) $$

根据作用量原理  $\delta S=0$ 分别证明:(1)当取弧长作为参量时,能够导出测地线方程:

$$ \frac{dx^{\mu}}{ds}\nabla_{\mu}\frac{dx^{\nu}}{ds}=0,\qquad(2) $$

其中,  $\nabla_{\mu}$  为无挠且与度规相容的协变导数,满足  $\Gamma_{[\alpha\beta]}^{\mu}=0$  以及  $\nabla_{\mu}g_{\alpha\beta}=0$。(2)请写出当采取任意参数时,测地线方程的形式。
\end{problem}

\begin{solution}
我先求解任意参数情况下的测地线方程,然后再代入弧长参数条件$ds^2 = g_{\mu\nu}dx^\mu dx^\nu$得到弧长参数下的测地线方程。
任意参数 $\lambda$ 下的弧长为:
\[
S = \int_{\lambda_i}^{\lambda_f} L \, d\lambda, \quad \text{其中} \quad L = \sqrt{g_{\mu\nu} \frac{dx^\mu}{d\lambda} \frac{dx^\nu}{d\lambda}}
\]
定义 $\dot{x}^\mu = \frac{dx^\mu}{d\lambda}$,则 $L = \sqrt{g_{\mu\nu} \dot{x}^\mu \dot{x}^\nu}$。根据作用量原理 $\delta S = 0$,得到欧拉-拉格朗日方程:
\begin{align}
\frac{d}{d\lambda} \left( \frac{\partial L}{\partial \dot{x}^\mu} \right) - \frac{\partial L}{\partial x^\mu} = 0 \label{eq:E-L}
\end{align}
此处度规是计算偏导数:
\begin{align}
\frac{\partial L}{\partial \dot{x}^\mu} &= \frac{1}{2L} \cdot 2 g_{\mu\nu} \dot{x}^\nu = \frac{g_{\mu\nu} \dot{x}^\nu}{L}, \label{eq:partialLdot} \\
\frac{\partial L}{\partial x^\mu} &= \frac{1}{2L} \partial_\mu g_{\alpha\beta} \dot{x}^\alpha \dot{x}^\beta \label{eq:partialLx}
\end{align}
代入方程 \eqref{eq:E-L} 得:
\begin{align}
\frac{d}{d\lambda} \left( \frac{g_{\mu\nu} \dot{x}^\nu}{L} \right) - \frac{1}{2L} \partial_\mu g_{\alpha\beta} \dot{x}^\alpha \dot{x}^\beta = 0 \label{eq:EL_arbitrary}
\end{align}
令 $u^\mu = \dot{x}^\mu$,则方程可写为:
\begin{align}
\frac{d}{d\lambda} \left( \frac{g_{\mu\nu} u^\nu}{L} \right) = \frac{1}{2L} \partial_\mu g_{\alpha\beta} u^\alpha u^\beta
\end{align}
展开左边:
\begin{align}
\frac{d}{d\lambda} \left( \frac{g_{\mu\nu} u^\nu}{L} \right) = \frac{1}{L} \frac{d}{d\lambda} (g_{\mu\nu} u^\nu) - \frac{g_{\mu\nu} u^\nu}{L^2} \frac{dL}{d\lambda}
\end{align}
其中,
\begin{align}
\frac{d}{d\lambda} (g_{\mu\nu} u^\nu) = \partial_\alpha g_{\mu\nu} u^\alpha u^\nu + g_{\mu\nu} \frac{du^\nu}{d\lambda}
\end{align}
代入后乘以 $L$:
\begin{align}
\partial_\alpha g_{\mu\nu} u^\alpha u^\nu + g_{\mu\nu} \frac{du^\nu}{d\lambda} - \frac{g_{\mu\nu} u^\nu}{L} \frac{dL}{d\lambda} - \frac{1}{2} \partial_\mu g_{\alpha\beta} u^\alpha u^\beta = 0
\end{align}
重组项:
\begin{align}
g_{\mu\nu} \frac{du^\nu}{d\lambda} + \left( \partial_\alpha g_{\mu\nu} u^\alpha u^\nu - \frac{1}{2} \partial_\mu g_{\alpha\beta} u^\alpha u^\beta \right) - \frac{g_{\mu\nu} u^\nu}{L} \frac{dL}{d\lambda} = 0 \label{eq:intermediate}
\end{align}
对于哑指标$\alpha$、$\beta$和$\nu$,通过更改指标并写为对称形式,括号内项可表示为:
\begin{align}
\partial_\alpha g_{\mu\nu} u^\alpha u^\nu - \frac{1}{2} \partial_\mu g_{\alpha\beta} u^\alpha u^\beta 
= \frac{1}{2} \left( \partial_\beta g_{\mu\alpha} + \partial_\alpha g_{\mu\beta} - \partial_\mu g_{\alpha\beta} \right) u^\alpha u^\beta
\end{align}

用$g^{\sigma\mu}$ 收缩方程 \eqref{eq:intermediate},根据克里斯托费尔联络的定义可得:
\begin{align}
\frac{du^\sigma}{d\lambda} + \Gamma_{\alpha\beta}^\sigma u^\alpha u^\beta - \frac{u^\sigma}{L} \frac{dL}{d\lambda} = 0
\end{align}

即:
\begin{align}
\frac{d^2 x^\sigma}{d\lambda^2} + \Gamma_{\alpha\beta}^\sigma \frac{dx^\alpha}{d\lambda} \frac{dx^\beta}{d\lambda} = \frac{1}{L} \frac{dL}{d\lambda} \frac{dx^\sigma}{d\lambda} \label{eq:geodesic_arbitrary}
\end{align}

令 $f(\lambda) = \frac{d}{d\lambda} \ln L$,则任意参数下的测地线方程为:
\begin{align}
\frac{d^2 x^\sigma}{d\lambda^2} + \Gamma_{\alpha\beta}^\sigma \frac{dx^\alpha}{d\lambda} \frac{dx^\beta}{d\lambda} = f(\lambda) \frac{dx^\sigma}{d\lambda} \label{eq:arbitrary_final}
\end{align}
用协变导数形式可写为:
\begin{align}
  \boxed{
\frac{dx^\alpha}{d\lambda} \nabla_\alpha \frac{dx^\sigma}{d\lambda} = f(\lambda) \frac{dx^\sigma}{d\lambda}
  }
\end{align}

现在代入弧长参数条件。当参数为弧长 $s$ 时,有$L = 1$
于是 $f(s)  = 0$。代入方程 \eqref{eq:arbitrary_final} 得:
\begin{align}
\frac{d^2 x^\sigma}{ds^2} + \Gamma_{\alpha\beta}^\sigma \frac{dx^\alpha}{ds} \frac{dx^\beta}{ds} = 0 \label{eq:geodesic_arc}
\end{align}
这就是弧长参数下的测地线方程。用协变导数表示:
\begin{align}
  \boxed{
  \frac{dx^\alpha}{ds} \nabla_\alpha \frac{dx^\sigma}{ds} = 0
  }
\end{align}
\end{solution}

\end{document}
